\documentclass{acm_proc_article-sp}
\usepackage{graphicx}
\usepackage{balance} 
\usepackage{multirow}
\usepackage{listings}
\usepackage{url}
\usepackage[usenames,dvipsnames]{color}
\begin{document}

\title{decimalInfinite: A Lossless, Order-Preserving Binary Encoding Of All Decimals}

\numberofauthors{1}

\author{
\alignauthor
Ghislain Fourny\\
       \affaddr{28msec, Inc.}\\
       \affaddr{Z\"urich, Switzerland}\\
       \email{g@28.io}
}
\date{6 March 2015}
\maketitle

\begin{abstract}
This paper introduces a binary encoding that supports arbitrarily large, small and precise decimals. It completely preserves information and order, and finally does not rely on any arbitrary use-cased-based choice of calibration.
\end{abstract}

\section{introduction}
In the early stages of computers, storage was scarce. And it was even scarcer in the processor itself. The first computers supported basic arithmetic operations on 8-bit integers. As more memory was needed, the size of the registers in the processor architecture was increased to successively 16 bits, 32 bits and very recently to 64 bits.

While this was mostly driven by the size of the memory, which never failed to exceed the maximum value space at some point, this also had an impact on the size of integers and decimals processed. This is simply because, in the Von-Neumann architecture \cite{VonNeumann1946}, registers are both used to store memory addresses and numbers.

The numbers used in programs are of two main kinds: integers and decimals. Integers are typically supported within a complete interval such as $- 2^{31}$ to $2^{31} -1$. In the case of decimals, it is a bit more complicated. They are limited both in range (driven by the size of their exponent) and in their precision.

For scientists that need to use bigger numbers and/or more precision, special libraries are available, and domain-specific software such as Maple \cite{MAPLE}, Mathematica \cite{MATHEMATICA} or Matlab \cite{MATLAB} make it possible to overcome the limitations of processors and shield the user from the complex machinery that is required to have the processor perform the computations.

In the modern database era, notably document stores, syntaxes such as XML \cite{XML} and JSON \cite{JSON}, but also their data models and the associated query languages, do not impose any limitation on integers or decimals: on the logical level, the entire value space is covered and the limit is only the size of the available storage.

The difference with scientific software, though, is that there is more to it than performing computations. These numbers need to be efficiently stored and retrieved, but most importantly, this must be done in such a way that they can be indexed. For efficiency reasons, it is desirable that this comparison can be done without decoding them. While this is trivial for hash indices, it becomes more difficult for range indices. Indeed, the lexicographical order of encoded values must correspond to the ordering of the corresponding numbers.

To the extent of our knowledge and our investigations, state-of-the-art databases, still, do not support yet such an encoding that would cover the entire decimal space (that is, the entire value space of the XML xs:decimal type, or the entire value space of JSON numbers). However, many encodings exist that have some of the desired properties. decimalInfinite unifies the ideas in these encodings in such a way that all these properties apply simultaneously.

This paper contributes an encoding that solves this problem. The encoding and decoding algorithms have been implemented in C++ and are used on production machines to store decimals as BSON binaries \cite{BSON}, on a MongoDB \cite{MongoDB} data store.

\subsection{Outline of the encoding}

The encoding relies on decomposing the underlying decimal into a normal form, as is typically done in other encodings. This decomposition is made of an overall sign, an exponent sign, an exponent and a significand (mantissa). These four components are encoded in turn and in this order.

Section \ref{section-related-work} gives an overview of the integer and decimal encoding landscape.
Section \ref{section-encoding} gives the algorithm for encoding decimals in the decimalInfinite format.
Section \ref{section-decoding} gives the decoding algorithm.
Section \ref{section-proof} gives a detailed proof of the main property of decimalInfinite (that it's order preserving).
Section \ref{section-fine-tune} shows how the core of the encoding can be extended to special numbers (NaN, infinity, etc).
Section \ref{section-complexity} briefly summarizes complexity aspects.

\section{Related work}
\label{section-related-work}

\subsection{Natural encoding}

Positive integers have a natural encoding which is simply their representation in base 2, as shown on figure \ref{figure-natural-encoding}.

\begin{figure}
\caption{Natural encoding of an integer (base 2)}
\label{figure-natural-encoding}
\center
\begin{tabular}{|l|l|}
\hline
Integer & Binary representation \\
\hline
1 & 1 \\
\hline
2 & 10 \\
\hline
3 & 11 \\
\hline
4 & 100  \\
\hline
5 & 101  \\
\hline
... & ... \\
\hline
\end{tabular}
\end{figure}

\begin{figure}
\caption{Natural encoding of an integer (base 2, padded to 4 bits), which is order preserving}
\label{figure-natural-encoding-padded}
\center
\begin{tabular}{|l|l|}
\hline
Integer & Binary representation (padded with 0s) \\
\hline
1 & 0001 \\
\hline
2 & 0010 \\
\hline
3 & 0011 \\
\hline
4 & 0100 \\
\hline
5 & 0101 \\
\hline
... & ... \\
\hline
\end{tabular}
\end{figure}

It supports an unlimited range, however is not order-preserving.

If however the range is limited, for example to 8, 16, 32 or 64 bits, as is commonly done in most programming languages, and the encodings are padded with leading 0s, then this encoding becomes order-preserving.

\subsection{Signed Integers}

Signed integers are commonly stored by encoding the sign in the first bit, meaning that positive integers (beginning with a 0, which is half the range) are stored in the same way as the unsigned encoding, while negative integers are stored beginning with a 1. Lexicographical order is only preserved for positive integers, for negative integers, but not overall.

\begin{figure}
\caption{Natural encoding of a signed integers (base 2, padded to 4 bits)}
\label{figure-natural-signed-encoding}
\center
\begin{tabular}{|l|l|}
\hline
Integer & Signed binary representation \\
\hline
-3 & 1101 \\
\hline
-2 & 1110 \\
\hline
-1 & 1111 \\
\hline
0 & 0000 \\
\hline
1 & 0001 \\
\hline
2 & 0010 \\
\hline
3 & 0011 \\
\hline
... & ... \\
\hline
\end{tabular}
\end{figure}

\subsection{Elias Gamma Code}
\label{section-gamma-code}

Gamma codes are a variable length encoding that supports the entire non-negative integer range ($\mathbb{N}$). One of their main usage and motivation is that they are prefix codes, meaning that they can get concatenated to each other in a way that they can still be separated again unambiguously. Figure \ref{figure-gamma-encoding} shows how it looks like for the first integers.

\begin{figure}
\caption{The Gamma Code for the first non-negative integers}
\label{figure-gamma-encoding}
\center
\begin{tabular}{|l|l|l|}
\hline
Integer & Offset integer & Gamma \\
\hline
0 & 1 (1) & 1 \\
\hline
1 & 2 (10) & 01 0  \\
\hline
2 & 3 (11) & 01 1  \\
\hline
3 & 4 (100) & 001 00 \\
\hline
4 & 5 (101) & 001 01 \\
\hline
5 & 6 (110) & 001 10 \\
\hline
... & ... \\
\hline
\end{tabular}
\end{figure}

 The main idea is that a first sequence of 0s, terminated by a 1, encodes the length of the binary-representation of the integer. Then, the natural binary representation of the integer (most of the time offset by 1), without its leading 1, follows.

Since the number of 0s is identical to the number of digits that follow after the 1, it is possible to unambiguously deduct where the encoding stops from the number of leading 0s.

For example, 00110010 can be unambiguously separated into 00110 (5) and 010 (1).

The Gamma code in its original form is order preserving. However, simply inverting the first sequence of 0s and the terminating 1 solves the issue, as is shown on Figure \ref{figure-gamma-encoding-tweaked}.

\begin{figure}
\caption{A slight modification of the Gamma Code that makes it order-preserving}
\label{figure-gamma-encoding-tweaked}
\center
\begin{tabular}{|l|l|}
\hline
Integer & Gamma (order-preserving) \\
\hline
0 & 0 \\
\hline
1 & 10 0  \\
\hline
2 & 10 1  \\
\hline
3 & 110 00 \\
\hline
4 & 110 01 \\
\hline
5 & 110 10 \\
\hline
... & ... \\
\hline
\end{tabular}
\end{figure}

\subsection{BCD, Chen-Ho, DPD}

The Binary-coded decimal encoding (BCD) encodes digits encodes decimal numbers by encoding each digit in a group of 4 or 8 digits, so-called tetrades. Improvements include the Chen-Ho \cite{ChenHo} encoding, which manages to encode 3 digits on 10 bits. It has the nice particularity of being extremely efficient to process (no multiplications, no divisions), and of being friendly to decimal computations, however does not preserve order. The Densely packed decimal encoding \cite{DPD} is an improvement upon the ChenHo encoding.

\subsection{IEEE float and double encoding}

The IEEE 754 standard specifies a couple of standard, floating-point encoding for decimals more commonly known as float or double. It both supports a finite range of decimals (its length is fixed), and is not order-preserving. It has several variants (binary16, binary32, binary64, binary128, decimal32, decimal64, decimal128) depending on the length and on the way the mantissa is encoded. These encodings rely on DPD.

\subsection{IBM Patent}
The US Patent 7685214 (``Order-preserving encoding formats of floating-point decimal numbers for efficient value comparison'') , filed by IBM, solves the order-preserving issue, but with a finite-length encoding, which implies that it supports a finite range of decimals only.

One very interesting idea in this approach is that, if the sign is negative, the mantissa m is encoded as 10-m to preserve the order.

\subsection{The Rishe Encoding}

The Rishe Encoding \cite{Rishe1992} is the closest match to decimalInfinite we found in literature. It supports arbitrarily large, small and precise decimals, is compact, and is also compatible with a bitwise lexicographic comparison. However, it relies on the arbitrary choice of 128 intervals based on the use case. decimalInfinite does not rely on such a choice and scales up regardless of how large, small or precise decimals are.

\section{Encoding}
\label{section-encoding}
Let us now get into the details of the encoding itself. The general idea is any non-zero decimal can be expressed in a canonical scientific form with four components (sign, exponent sign, exponent, mantissa). These four components can be encoded separately and concatenated. Since each of the components (but the last one) is a prefix code, it can be unambiguously decoded again.

\subsection{Encoding of zero}

Zero is handled separately and encoded as $10$.

\subsection{Canonical decomposition}

Any decimal number, except zero, can be expressed uniquely in scientific notation as in commonly done in literature, that is, in the form

$$s\times m \times10^{t\times e}$$

where:

\begin{itemize}
\item The overall sign is $s\in \{-1, 1\}$.
\item The exponent $e\in \mathbb{N}$ is a non-negative integer (which is the absolute value of the logarithm in base ten of the original number rounded down to the next integer).
\item The exponent sign is $t\in \{-1, 1\}$.
\item The mantissa is $m\in [1,10)$, a real number between 1 (included) and 10 (excluded).
\end{itemize}

If S denotes the encoding s, T that of t and so on, then the overall encoding comes naturally as STEM as shown on figure \ref{figure-overall-encoding}. This is because decimal numbers in scientific notation can be sorted with the following criteria in this order:
\begin{enumerate}
\item sign
\item exponent sign
\item exponent
\item mantissa
\end{enumerate}

\begin{figure}
\caption{Encoding of an overall decimal in scientific notation $s\times m \times10^{t\times e}$}
\label{figure-overall-encoding}
\center
\begin{tabular}{|l|l|l|l|}
\hline
$S$ & $T$ & $E$ & $M$ \\
\hline
\end{tabular}
\end{figure}

Throughout this paper, four examples, which cover various combinations of the four components, will be used:

$$-103.2 = - 1.032 \times 10^2$$

$$-0.0405 = -4.05 \times 10^{-2}$$

$$0.707106 = 7.07106 \times 10^{-1}$$

$$4005012345 = 4.005012345 \times 10^9$$


\subsection{Encoding the sign}

The sign of a decimal is encoded on two bits as shown on Figure \ref{figure-sign}

\begin{figure}
\caption{Encoding of the overall decimal sign}
\label{figure-sign}
\center
\begin{tabular}{|l|l|}
\hline
S & Sign s \\
\hline
00 &  negative sign ($s=-1$, e.g., $-4.3\times10^3$)\\
\hline
10 & positive sign ($s=1$, e.g., $4.3\times10^3$)\\
\hline
\end{tabular}
\end{figure}

Since, zero is simply encoded with $10$ with no further bits, it is already apparent that its encoding appears lexicographically after the encoding of any negative decimal, and before the encoding of any positive decimal.

The reason for using two bits rather than just one is that negative infinity (-INF), positive infinity (+INF) as well as negative zero and NaN can be conveniently encoded as well (see Figure \ref{figure-sign-extended}).

So far, our four example have an encoding that begins as follow:

\begin{tabular}{l|l}
$- 1.032 \times 10^2$ & 00... \\

$-4.05 \times 10^{-2}$ & 00... \\

$0$ & 10 \\

$7.07106 \times 10^{-1}$ & 10... \\

$4.005012345 \times 10^9$ & 10...\\
\end{tabular}
\subsection{Encoding the exponent sign}

The exponent sign is encoded by either negating or not negating all bits in the encoding of the absolute value of the exponent (Section \ref{section-exponent-encoding}), in such a way that the third bit be a 0 or a 1 as shown on figure \ref{figure-exponent-sign}.

\begin{figure}
\caption{Encoding of the exponent sign}
\label{figure-exponent-sign}
\center
\begin{tabular}{|l|l|}
\hline
S and T & s and t \\
\hline
00 0 &  negative sign, positive exponent sign\\
\hline
00 1 &  negative sign, negative exponent sign\\
\hline
10 0 & positive sign, negative exponent sign\\
\hline
10 1 & positive sign, positive exponent sign\\
\hline
\end{tabular}
\end{figure}

The encoding of our four examples continues as follows:

\begin{tabular}{l|l}
$- 1.032 \times 10^2$ & 00 0... \\

$-4.05 \times 10^{-2}$ & 00 1... \\

$7.07106 \times 10^{-1}$ & 10 0... \\

$4.005012345 \times 10^9$ & 10 1...\\
\end{tabular}

\subsection{Encoding the exponent}

The absolute value of the exponent is encoded with a modified gamma code (as explained in section \ref{section-gamma-code}), using an offset of 2.

\label{section-exponent-encoding}
\begin{enumerate}
\item The exponent is offset by +2, for example, 4 is encoded with the modified gamma code of 6. 
\item  The offset exponent is written in a binary form, for example, 6 is written 110.
\item  Call N the number of its digits (in the case of 110: 3).
\item  The first digit is replaced with N-1 ones, followed by a zero (in the case of 110: 110 10)
\end{enumerate}

Figure \ref{figure-exponent-encoding} shows how the smallest absolute values of the exponent are encoded. If the overall sign and the exponent sign are identical, this is how it appears in the final encoding. Otherwise, all the bits are negated.

\begin{figure}
\caption{A slight modification of the Gamma Code that makes it order-preserving}
\label{figure-exponent-encoding}
\center
\begin{tabular}{|l|l|l|}
\hline
Integer & Integer offset by 2 & Exponent encoding\\
\hline
0 & 2 (10) & 10 0 \\
\hline
1 & 3 (11) & 10 1  \\
\hline
2 & 4 (100) & 110 00  \\
\hline
3 & 5 (101) & 110 01 \\
\hline
4 & 6 (110) & 110 10 \\
\hline
5 & 7 (111) & 110 11 \\
\hline
6 & 8 (1000) & 1110 000 \\
\hline
7 & 9 (1001) & 1110 001 \\
\hline
8 & 10 (1010) & 1110 010 \\
\hline
9 & 11 (1011) & 1110 011 \\
\hline
... & ... & ...\\
\hline
\end{tabular}
\end{figure}

The encoding of our four examples continues as follows:

\begin{tabular}{l|l}
$- 1.032 \times 10^2$ ($e=2$, opposite signs) & 00 001 11... \\

$-4.05 \times 10^{-2}$ ($e=2$, same signs) & 00 110 00... \\

$7.07106 \times 10^{-1}$ ($e=1$, opposite signs) & 10 01 0... \\

$4.005012345 \times 10^9$ ($e=9$, same signs) & 10 1110 011...\\
\end{tabular}


\subsection{Encoding the mantissa}

The mantissa is encoded in a way similar to decimal32, decimal64 and decimal128, that is:

\begin{itemize}
\item its initial digit (before the comma) is encoded on 4 bits in its natural binary representation.
\item the remaining digits (after the comma) are grouped 3 by 3 (declets). Each declet is encoded in its natural binary representation on 10 bits. Trailing 0s are added to make sure that the last group also has 3 digits.
\end{itemize}

If the overall sign of the decimal is negative though, a trick similar to the IBM patent is used: 10-m is encoded instead of m (in this case, the leading digit may be a 0 and will never be a 9).

\begin{figure}
\caption{Examples of mantissa encodings}
\label{figure-mantissa-encoding}
\center
\begin{tabular}{|l|l|l|l}
\hline
$8.968 (=10-1.023)$ & 8 968 & 1000 \\
& & 1111001000 \\
\hline
$5.95 (=10-4.05)$ & 5 950 & 0101 \\
& & 1110110110\\
\hline
$7.07106$ & 7 071 060 & 0111 \\
& & 0001000111\\
& & 0001111000\\
\hline
$4.005012345$ & 4 005 012 345 & 0100 \\
& & 0000000101\\
& & 0000001100\\
& & 0101011001\\
\hline
\end{tabular}
\end{figure}

The encoding of our four examples can now be completed:

\begin{tabular}{l}
$- 1.032 \times 10^2$ (10-m is taken)\\
00 001 11 1000 1111001000\\
\\
$-4.05 \times 10^{-2}$ (10-m is taken)\\
00 110 00 0101 1110110110\\
\\
$7.07106 \times 10^{-1}$\\
10 01 0 0111 0001000111 0001111000\\
\\
$4.005012345 \times 10^9$\\
10 1110 011 0100 0000000101 0000001100 0101011001\\
\end{tabular}

\section{Decoding}
\label{section-decoding}

Decoding is also performed from left to right, in a way similar to encoding.

\subsection{Decoding the overall sign}

The overall decimal sign is obtained straightforwardly from the first two bits. If no more bits follow, it's a zero. Otherwise, decoding continues with the exponent.

\subsection{Decoding the exponent}

\subsubsection{Decoding the exponent sign}

The exponent sign can be deduced from the third bit, but depends on the overall sign:

\begin{itemize}
\item If the overall sign is - and the third bit is a 0, the exponent sign is +.
\item If the overall sign is - and the third bit is a 1, the exponent sign is -.
\item If the overall sign is + and the third bit is a 0, the exponent sign is -.
\item If the overall sign is + and the third bit is a 1, the exponent sign is +.
\end{itemize}

\subsubsection{Decoding the exponent}

The exponent encoding, starting at the third bit, is of variable length. Since gamma codes are prefix codes though, determining the length of the exponent encoding is straight forward.

One starts at the third bit and, including it, counts the number of identical bits that follow. If there is a sequence of N identical bits (whether 0s or 1s) starting from the third bit, then the exponent is encoded on 2N+1 bits.

An example best illustrates this.

1011100110100000000010100000011000101011001.

Starting from the third bit, there is a sequence of three 1s, so the exponent is on 7 bits

10 \textbf{1110011} 0100000000010100000011000101011001.

The next step is to flip all the bits in the exponent encoding if the leading bit is a 0. In the example, no change is needed.

The exponent is then decoded as a modified gamma code (Section \ref{section-gamma-code}) and offset by -2:

\begin{enumerate}
\item The first N+1 bits are replaced with a 1 (in the example: 1011).
\item The obtained bit sequence is decoded as a natural binary representation (11).
\item One substracts 2 (example: 9).
\end{enumerate}

\subsection{Decoding the mantissa}

The mantissa is decoded in groups of 10 bits (except the first group which has 4 bits). Each group is decoded as a natural binary representation. The first group gives the digit before the comma, the other groups give the digits (three per group) after the comma.

If the first group does not deliver a number comprised between 0 and 9, or a subsequent group does not deliver a number comprised between 0 and 999, an error is raised.

Finally, if the overall sign is negative, the complement to 10 is taken instead.

\section{Why it is order-preserving}
\label{section-proof}

The encoding are designed in such a way that, if $a < b$, then the encoded $a$ ($A$) comes lexicographically before ($<<$) the encoded $b$ ($B$).

A proof thereof now follows.

$a$'s (absolute) exponent is called $c$, $b$'s exponent is called $d$. $a$'s mantissa is called $e$, b's mantissa is called $f$.
 
\begin{enumerate}
\item If $a$ is negative and $b$ is positive, then $A$ begins with 00 and $B$ with 10, so that $A << B$.
\item If $a$ and $b$ are both positive, then they both begin with 10.
\begin{enumerate}
  \item If $A$'s exponent is negative and $B$'s exponent is positive, the next digit in $A$ will be a 0 and that of $B$ a 1, so that $A << B$.
  \item If $A$'s exponent and $B$'s exponent are both positive, and $c+2$ has less digits than $d+2$, then $A$ will have less 1s than $B$ in front of the next 0, so that $A << B$.
  \item If $A$'s exponent and $B$'s exponent are both positive but different, then $c < d$ and $c+2$ has as many digits ($N$) as $d+2$ then both $A$ and $B$ will both have $N-1$ 1s followed by a 0. The next $N-1$ digits after the 0 in $A$ and $B$ correspond to a natural binary representation (with no leading 1) of $c$ and $d$ respectively, so that $A << B$ because the natural binary representations preserve order given a fix number of digits.
  \item If $A$'s exponent and $B$'s exponent are both negative but different, then $c > d$ and $c+2$ has as many digits ($N$) than $d+2$ then both $A$ and $B$ will both have $N-1$ 0s followed by a 1. The next $N-1$ digits after the 0 in $A$ and $B$ ($E$ and $F$ respectively) correspond to an inverted natural binary representation, with no leading 0, of $c$ and $d$ respectively. Since $c > d$, $E << F$ because it's inverted, and $A << B$.
  \item If $A$'s exponent and $B$'s exponent are equal, then $e < f$ and then both $A$ and $B$ have the same next $2N-1$ digits. The next digits of $A$ and $B$, organized in one group of 4, then groups of 10, are all natural binary representations of $e$ and $f$ and preserve the order, so that $A << B$.
  \end{enumerate}
\item If $A$ and $B$ are both negative, then both begin with 01.
  \begin{enumerate}
  \item If $A$'s exponent is positive and $B$'s exponent is negative, the next digit in $A$ will be a 0 and that of $B$ a 1, so that $A << B$.
  \item If $A$'s exponent and $B$'s exponent are both negative but different, then $c < d$ and $c+2$ has as many digits ($N$) than $d+2$ then both $A$ and $B$ will both have $N-1$ 1s followed by a 0. The next $N-1$ digits after the 0 in $A$ and $B$ correspond to a natural binary representation (with no leading 1) of $c$ and $d$ respectively, so that $A << B$ because the natural binary representations preserve order given a fix number of digits.
  \item If $A$'s exponent and $B$'s exponent are both positive but different, then $c > d$ and $c+2$ has as many digits ($N$) than $d+2$ then both $A$ and $B$ will both have $N-1$ 0s followed by a 1. The next $N-1$ digits after the 0 in $A$ and $B$ ($E$ and $F$ respectively) correspond to an inverted natural binary representation, with no leading 0, of $c$ and $d$ respectively. Since $c > d$, $E << F$ because it's inverted, and $A << B$.
  \item If $A$'s exponent and $B$'s exponent are equal, then $e > f$ and both $A and B$ have the same next $2N-1$ digits. The next digits of $A$ and $B$, organized in one group of 4, then groups of 10, are all natural binary representations of $10-e$ and $10-f$ and preserve the order. Since $10-e < 10-f$, $A << B$.
  \end{enumerate}
\item -0 is encoded as 10 and 10 is lexicographically smaller than the encodings of negative decimals, which begin with 01. it is lexicographically greater than the encodings of positive decimals, which begin with 10 followed by at least one further digit.
\end{enumerate}

\section{Fine-tuning the scheme}
\label{section-fine-tune}

\subsection{Special numbers}

Special numbers such as positive and negative infinity, negative zero can also be encoded in such a way that the lexicographic order still holds, as shown on figure \ref{figure-sign-extended}. NaN can also be encoded (even if the order does not apply in this case).

\begin{figure}
\caption{Adding special numbers}
\label{figure-sign-extended}
\center
\begin{tabular}{|l|l|}
\hline
00 & -INF \\
\hline
00... &  negative sign (e.g., $-4.3\times10^3$)\\
\hline
01 & negative zero \\
\hline
10 & positive zero \\
\hline
10... & positive sign (e.g., $4.3\times10^3$)\\
\hline
11 & +INF \\
\hline
111 & NaN \\
\hline
\end{tabular}
\end{figure}

\subsection{Trailing zeros}

To save space, trailing 0s can be removed from the binary encoding and added back while decoding (to fit the size of the last declet group).

\subsection{Fix-length variant}

In environments where encoding preserving lexicographic order is not supported across different lengths (this is the case with MongoDB's ordering of binaries), this encoding can be adapted to work at the cost of limiting the range.
A prefix of the binary encoding can be taken as an approximation of the encoded decimal, possibly padded with leading 0s if too short. This works as long as the total stored lengths exceeds the length of the encoding of the sign and exponent, which limits the range.

\section{Complexity}
\label{section-complexity}
The encoding and decoding time, as well as the storage space, is linear in the size (number of digits) of the mantissa, and linear in the logarithm of the exponent, that is, in the logarithm of the logarithm of the decimal.

The encoding of the mantissa uses a very common approach that is very compact in terms of entropy, and making it more compact (for example, by grouping bits in bigger groups) would increase computational. The encoding of the exponent deviates from an optimal size by a constant factor of 2, which is the cost of using the Gamma prefix code.

\section{Conclusion}

We introduced a binary encoding that:
\begin{itemize}
\item supports the entire decimal value space including special numbers
\item does not lose precision
\item preserves the order, in the sense that the encoding is a homomorphism between the decimals (sorted naturally) and the bit sequences (sorted lexicographically).
\end{itemize}

The encoding and decoding algorithms were successfully implemented in C++, and is used to store JSON numbers lossless in BSON binaries on an underlying MongoDB layer. Unfortunately, MongoDB does not use a full lexicographical ordering of binaries (rather, it first sorts by size and then lexicographically), such that padding to a fixed-length is needed for this vendor.

The code has been running with no known issues on production servers since 2012.

\section{Acknowledgements}
decimalInfinite was implemented on the 28msec platform with the help of Matthias Brantner and Till Westmann.

\bibliographystyle{abbrv}
\bibliography{decimalinfinite}

\end{document}
