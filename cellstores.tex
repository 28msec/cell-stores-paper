% THIS IS AN EXAMPLE DOCUMENT FOR VLDB 2012
% based on ACM SIGPROC-SP.TEX VERSION 2.7
% Modified by  Gerald Weber <gerald@cs.auckland.ac.nz>
% Removed the requirement to include *bbl file in here. (AhmetSacan, Sep2012)
% Fixed the equation on page 3 to prevent line overflow. (AhmetSacan, Sep2012)

\documentclass{vldb}
\usepackage{graphicx}
\usepackage{balance}  % for  \balance command ON LAST PAGE  (only there!)


\begin{document}

% ****************** TITLE ****************************************

\title{Cell Stores}

% possible, but not really needed or used for PVLDB:
%\subtitle{[Extended Abstract]
%\titlenote{A full version of this paper is available as\textit{Author's Guide to Preparing ACM SIG Proceedings Using \LaTeX$2_\epsilon$\ and BibTeX} at \texttt{www.acm.org/eaddress.htm}}}

% ****************** AUTHORS **************************************

% You need the command \numberofauthors to handle the 'placement
% and alignment' of the authors beneath the title.
%
% For aesthetic reasons, we recommend 'three authors at a time'
% i.e. three 'name/affiliation blocks' be placed beneath the title.
%
% NOTE: You are NOT restricted in how many 'rows' of
% "name/affiliations" may appear. We just ask that you restrict
% the number of 'columns' to three.
%
% Because of the available 'opening page real-estate'
% we ask you to refrain from putting more than six authors
% (two rows with three columns) beneath the article title.
% More than six makes the first-page appear very cluttered indeed.
%
% Use the \alignauthor commands to handle the names
% and affiliations for an 'aesthetic maximum' of six authors.
% Add names, affiliations, addresses for
% the seventh etc. author(s) as the argument for the
% \additionalauthors command.
% These 'additional authors' will be output/set for you
% without further effort on your part as the last section in
% the body of your article BEFORE References or any Appendices.

\numberofauthors{1} %  in this sample file, there are a *total*
% of EIGHT authors. SIX appear on the 'first-page' (for formatting
% reasons) and the remaining two appear in the \additionalauthors section.

\author{
% You can go ahead and credit any number of authors here,
% e.g. one 'row of three' or two rows (consisting of one row of three
% and a second row of one, two or three).
%
% The command \alignauthor (no curly braces needed) should
% precede each author name, affiliation/snail-mail address and
% e-mail address. Additionally, tag each line of
% affiliation/address with \affaddr, and tag the
% e-mail address with \email.
%
% 1st. author
\alignauthor
Ghislain Fourny\\
       \affaddr{28msec, Inc.}\\
       \affaddr{Z\"urich, Switzerland}\\
       \email{g@28.io}
}
% There's nothing stopping you putting the seventh, eighth, etc.
% author on the opening page (as the 'third row') but we ask,
% for aesthetic reasons that you place these 'additional authors'
% in the \additional authors block, viz.
%\additionalauthors{Additional authors: John Smith (The Th{\o}rv\"{a}ld Group, {\texttt{jsmith@affiliation.org}}), Julius P.~Kumquat
%(The \raggedright{Kumquat} Consortium, {\small \texttt{jpkumquat@consortium.net}}), and Ahmet Sacan (Drexel University, {\small \texttt{ahmetdevel@gmail.com}})}
\date{8 September 2014}
% Just remember to make sure that the TOTAL number of authors
% is the number that will appear on the first page PLUS the
% number that will appear in the \additionalauthors section.


\maketitle

\begin{abstract}
false
\end{abstract}


\section{Introduction}
In 1970, Codd [cite] introduced the relational model as an alternative to the graph and network models (such as file systems) in order to provide a more suitable interface to users, and to protect them from internal representations (``data independence'').

The relational model's first implementation was made public in 1976 by IBM [cite].

In the last four decades, the relational model has been enjoying undisputed popularity and has been widely used in enterprise environments. This is probably because it is both very simple to understand and universal. Furthermore, it is accessible to business users without IT knowledge, to whom tabular structures are very natural --- as demonstrated by the strong usage of spreadsheet software (such as Microsoft Excel) as well as user-friendly front-ends (such as Microsoft Access).

However, in the years 2000s, the exponential explosion of the quantity of data to deal with increasingly showed the limitations of this model. Several companies, such as Google, Facebook or Twitter needed scaling up and out beyond the capabilities of any RDBMS, both because of the \emph{quantity} of data (rows), and because of the \emph{high dimensionality} of this data (columns). Each of them built their own, ad-hoc data management system (Big Table, Cassandra, ...). These technologies often share the same design ideas (scale up through clustering and replication, high dimensionality handling through data heterogeneity and tree structures), which led to the popular common denomination of NoSQL, a common roof for:
\begin{itemize}
\item key-value stores
\item document stores
\item column stores
\item graph databases
\end{itemize}

NoSQL solves the scale-up issue, but at a two-fold cost:
\begin{itemize}
\item for developers, the level of abstraction provided by NoSQL stores is much lower than that of the relational model. These data stores often provide limited querying capability such as point or range queries, insert, delete and update (CRUD). Higher-level operations such as joins must be implemented in a host language, on a case by case basis.
\item for business users, these data models are much less natural than tabular data. Reading and editing data formats such as XML, JSON requires at least basic IT knowledge. Furthermore, business users should not have to deal with indexes at all.
\end{itemize}

This is a major step back from Codd's intentions back in the 1970s, as the very representations he wanted to protect users from (tree-like data structures, storage, ...) are pushed back to the user.

Reluctance can be observed amongst users, and this might explain why the ``big three'' (Oracle, Microsoft, IBM) are heavily forcing the usage of the SQL language on top of these data stores.

This paper introduces the cell stores data paradigm, whose goal is to (i) leverage the technological advancements made in the last decade, while (ii) bringing back to business users control and understanding over their data.

Cell stores are at a sweet spot between on the one hand key-value stores, in that they scale up seamlessly and gracefully in the quantity of data as well as the dimensionality of the data, and on the other end the relational model, in that business users access the data in tabular views via familiar, spreadsheet-like interfaces.

Section... gives an overview of state of the art technologies for storing large quantities of highly dimensional data and their shortcomings. Section ... motivates the need for the cell stores paradigm. Section ... introduces the data model behind cell stores. Section ... shows how a relational database can be stored naturally in a cell store. Section... points out that there is a standard format for exchanging data between cell stores as well as other databases. Section... gives implementation-level details. Section... gives a few more miscellaneous remarks, and section... explores performance.

\section{State of the art}
Relational databases

Key value stores

Document stores

Column stores

\section{Why cell stores}

Sparseness

User-friendliness

Cell stores bring back control to business users:

\begin{itemize}

\item scales up with heterogeneous data

\item accessible via spreadsheet-like interface

\item control of taxonomies (hierarchies of concepts)

\item business rules

\end{itemize}


\section{The Cell Store Data Model}

\subsection{Gas of cells}

\subsection{Hypercubes}

\subsection{Spreadsheet views}

\section{Canonical mapping to the relational model}

\section{Standardized Data Interchange: XBRL}

Facts, taxonomies

Concepts, dimensions, members, hypercubes

Formulas

Table linkbases

\section{Implementation}

\subsection{On top of a document store: NoLAP}

\subsection{On top of a column store (e.g. Big Table)}

\subsection{On top of a key-value store (Cassandra)}

\section{Miscellaneous}

\subsection{Temporal databases}

Transaction time

Valid time

\section{Performance}

secxbrl.info

\section{Conclusion}

\section{Acknowledgements}

This is joint work. The implementation of the first cell store on top of a document store has been made as a team effort by Matthias Brantner, William Candillon, Federico Cavalieri, Dennis Knochenwefel, Alexander Kreutz and myself.

% The following two commands are all you need in the
% initial runs of your .tex file to
% produce the bibliography for the citations in your paper.
\bibliographystyle{abbrv}
\bibliography{vldb_sample}  % vldb_sample.bib is the name of the Bibliography in this case
% You must have a proper ".bib" file
%  and remember to run:
% latex bibtex latex latex
% to resolve all references

%APPENDIX is optional.
% ****************** APPENDIX **************************************
% Example of an appendix; typically would start on a new page
%pagebreak

%\begin{appendix}
%You can use an appendix for optional proofs or details of your evaluation which are not %absolutely necessary to the core understanding of your paper. 

%\section{Final Thoughts on Good Layout}
%\end{appendix}



\end{document}
